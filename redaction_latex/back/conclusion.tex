\part*{Conclusion}
\addcontentsline{toc}{part}{Conclusion}
\markboth{Conclusion}{Conclusion}


Un des résultats méthodologiques forts de ce mémoire a d'abord été de montrer qu'un traitement automatisé des journaux de petites annonces est possible, notamment grâce à l'entraînement de modèles d'apprentissage pour l'extraction et la catégorisation d'entités, mais également en tirant profit de l'uniformité formelle des annonces pour employer des expressions régulières. 

Les incursions vers les différentes composantes des demandes d'emploi ont permis de montrer en quoi l'annonce n'est pas un support qui vient perturber de bout en bout la recherche de travail à l'époque moderne. Au contraire, elle perpétue par beaucoup d'aspects les processus et les structures du marché de l'emploi domestique aux XVIIIè-XIXè siècles: le marché des annonces reste majoritairement mixte, occupé par des domestiques peu qualifiés, jeunes et aux savoir-faire manuels. L'emploi reste conditionné à l'intervention de multiples intermédiaires et garants dans le processus de recrutement, dont l'annonce n'est finalement qu'une composante parmi d'autres. 

Mais, ponctuellement, ou selon certaines modalités bien précises, l'annonce peut s'éloigner de la norme domestique, et refléter les transformations qui accompagnent l'évolution de la condition et de la conjecture économique à la fin de l'Ancien Régime. Ainsi, j'ai pu montrer dans la troisième partie, en continuité avec les travaux d'Ulrike Krampl, que la domesticité des annonces n'est pas le reflet exact de la domesticité dans son ensemble, mais plutôt son miroir déformé: y sont surreprésentés des hommes domestiques qualifiés, professeurs, secrétaires ou régisseurs, pour qui le format des annonces s'inscrit en continuité directe de leur pratique quotidienne de l'écriture. Le lectorat des annonces, plutôt bourgeois, représente également une cible intéressante pour ces domestiques spécialisés, qui n'officient généralement quand dans les grandes maisons. 

L'étude du genre dans les annonces a également permis de mettre en évidence la reconduction sur le papier de certains processus déjà connus de la discipline historique: le cantonnement des femmes à certaines tâches domestiques, le contrôle social qui s'impose à elles et les obligent à mettre en avant leur probité devant d'éventuelles employeurs. 

Enfin, l'étude numérique des petites annonces a permis d'interroger les raisons de l'usage de ce support par les domestiques. Face à la subsistance des réseaux informels, pourquoi se donner la peine d'aller jusqu'au Bureau d'adresses, de (parfois) payer une annonce, puis se soumettre au hasard et aux potentiels risques de l'emploi par un maître inconnu? Si l'historiographie a pu soulever les hypothèses très probables de la saturation du marché traditionnel du travail\footcites{pasleauAnnoncesEmploiLiege2002}{sarasuaCriadosNodrizasAmos1994} ou de la crise économique qui joue en faveur des maîtres et invite à multiplier les méthodes de recherche, je pense qu'il est également important d'envisager les avantages et les espaces de liberté que peuvent procurer les annonces, tout particulièrement à l'aune des difficultés structurelles que je viens de mentionner. Certaines déclarations des domestiques, notamment relatives aux bureaux de placement, doivent permettre de penser l'annonce comme espace émancipatoire, bien que hautement contraint, qui permet d'échapper en partie aux constrictions du placement. Pour les domestiques qui viennent d'arriver en ville, n'ont ni réseau ni voisins, ou qui souhaitent changer d'emploi sans le faire savoir à leur entourage, on peut également imaginer que l'annonce représente une possibilité bienvenue.
 
Cette étude du marché de la domesticité au moyen des méthodes numériques a ainsi permis d'obtenir plusieurs résultats intéressants aussi bien pour l'histoire du genre que du travail ou de la presse. Si certaines problématiques (les questions des offres d'emploi, des domestiques étrangers, des voyages ou de la mobilité des annonces et des informations en dehors de la ville, pour ne citer qu'elles) ont été laissées de côté, ou traitées moins exhaustivement, j'espère malgré tout avoir pu montrer l'intérêt des méthodes numériques pour faire une histoire, non pas forcément massive, mais riche et socialement située de groupes et d'individus souvent écartés du récit historique. 




 