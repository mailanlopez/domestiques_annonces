\section*{Résumé}
\addcontentsline{toc}{chapter}{Résumé}

Ce mémoire prend pour objet les petites annonces d'emploi domestique publiées entre le milieu du XVIIIè siècle et le début du XIXè siècle dans trois villes françaises, Paris, Lyon et Bordeaux. À travers l'exploitation numérique de ces sources, il vise à étudier à la fois l'économie générale des annonces, leur contenu, le profil socio-démographique des travailleuses et travailleurs qui les publient, et les différentes stratégies qui différencient ce support du marché traditionnel du travail. La mobilisation de méthodes de traitement automatique des langues est également l'occasion de s'intéresser aux formes de publicité de soi et de son travail qui composent l'annonce, et qui permettent d'identifier ce qui constitue la "bonne domesticité" à l'époque moderne.

\medskip

\textbf{Mots-clés: histoire moderne; humanités numériques ; études sur le genre ; traitement automatique de la langue ; apprentissage machine ; travail ; domesticité ; presse; petites annonces.}

\textbf{Informations bibliographiques:} Mai Lan Lopez, \textit{Décrire son travail, écrire sa condition: femmes et hommes domestiques dans la presse d'annonces (Paris, Lyon, Bordeaux, 1750-1815)}, mémoire de master 2 \og Humanités Numériques\fg{}, dir. Clyde Plumauzille, Marie Puren, Université Paris, Sciences \& Lettres, 2023.



\section*{Abstract}

This dissertation focuses on domestic job advertisements published between the mid-18th century and the early 19th century in three French cities: Paris, Lyon, and Bordeaux. Through the digital study of these sources, it aims to better understand the overall economy of the advertisements, their content, the socio-demographic profiles of the workers who published them, and the various strategies that distinguish this type of media from the traditional labor market. The use of Natural Language Processing methods also provides an opportunity to explore the forms of self-promotion and self-representation within these advertisements, allowing us to identify what constituted "good domestic service" in the modern era.

\medskip

\textbf{Keywords: modern history; digital humanities; gender studies; natural language processing; machine learning; work; domesticity; press; personal ads.}

\textbf{Bibliographic information:} Mai Lan Lopez, \textit{Describing labor, inscribing work: women and men domestic servants in personal ads (Paris, Lyon, Bordeaux, 1750-1815)}, M.A. thesis \og Digital Humanities\fg{}, dir. Clyde Plumauzille, Marie Puren, Université Paris, Sciences \& Lettres, 2023.


\clearpage