\part*{Introduction}
\addcontentsline{toc}{part}{Introduction}
\markboth{Introduction}{Introduction}


Il parait difficile de concevoir, de notre point de vue contemporain, la place qu'a pu occuper la domesticité dans le paysage urbain et social français avant le XXè siècle. Beaucoup d'indices, néanmoins, mettent en exergue le rôle économique et culturel considérable des domestiques, à commencer par leur poids démographique: l'historiographie moderne considère qu'ils et elles représentent entre 5 et 15\% de la population selon les villes, des chiffres qui fluctuent beaucoup selon les conjonctures. À Paris, au XVIIIè siècle, Jacqueline Sabattier estime qu'ils et elles représentent environ 50 000 personnes, soit 10\% de la population, et que leur proportion tend à augmenter pendant le siècle. Une famille parisienne sur quatre emploierait alors au moins un ou une domestique\footcites[p.28]{mazaServantsMastersEighteenthcentury1983}.

Ombres et témoins de leurs maîtres, omniprésents au théâtre, dans la littérature et dans les traités de police mais invisibles ou presque dans les sources de la pratique et les archives du for privé, les domestiques ont pendant longtemps moins intéressé la discipline historique que d'autres catégories de population, mieux représentées dans les sources (compagnons, apprentis, ouvrières ...). Du fait de leur association avec la noblesse puis la bourgeoisie, les domestiques ont longtemps été tenus à l'écart de l'étude des populations laborieuses et dominées, dont le cœur a toujours été la condition ouvrière. 

Depuis la seconde moitié du XXè siècle, néanmoins, l'histoire démographique, puis l'histoire du travail, ont semblé ouvrir le grand chantier de l'histoire de la condition ancillaire. Portée notamment par des projets de recherche à l'échelle européenne\footcites{pasleauProceedingsServantProject2001}, celle-ci a fait l'objet dès la fin des années 1980 de synthèses qui font encore date aujourd'hui, notamment grâce aux statistiques qu'elles fournissent sur ce groupe peu connu\footcites{sabattierFigaroSonMaitre1984,guttonDomestiquesServiteursDans1981,fairchildsDomesticEnemiesServants1984, mazaServantsMastersEighteenthcentury1983}. À partir des sources de l'histoire sociale (notamment les insinuations testamentaires, les contrats de mariage et les déclarations de grossesse), la discipline a peu à peu contribué à la formation d'acquis historiographiques forts sur la domesticité, une population pourtant caractérisée par sa diversité interne. 

L'histoire du genre, enfin, s'est également emparée du sujet à partir des années 2000, en le retravaillant à l'aune de la féminisation croissante du secteur à partir de la fin du XVIIIè siècle, de l'association culturelle et matérielle des femmes au foyer et aux activités domestiques, puis avec l'outil théorique du \textit{care}. 

C'est nourrie de ces différentes approches et traditions historiographiques que je cherche dans ce mémoire à étudier la domesticité du XVIIIè siècle à partir d'une source peu mobilisée jusque-là par les historiennes et historiens du travail: la presse d'annonces. 


\section{Histoire de la presse d'annonces en France}

\subsection{La naissance des journaux d'annonces}

Le premier journal d'annonces français naît à Paris au XVIIè siècle\footcite{martinTroisSieclesPublicite1992}; il est à l'époque une émanation du Bureau d'adresses ouvert en 1630 par le médecin et proche du cardinal Richelieu Théophraste Renaudot, "pour le règlement général des pauvres". Moins de dix ans plus tard, une ordonnance royale oblige les travailleurs sans emplois à s'y présenter; le Bureau est lié, dès sa naissance, "à la grande entreprise de mise en ordre de la société française par la monarchie absolue naissante". Théophraste Renaudot obtient à la même époque le privilège de la \textit{Gazette}, qui deviendra le principal organe de presse du royaume et un outil de propagande essentiel pour les acteurs de la monarchie, Richelieu en tête.

C'est également peu après l'ouverture du Bureau que les \textit{Feuilles d'adresses} commencent à être diffusées, à l'époque tous les dix jours. Elles répertorient les demandes et les offres déposées au Bureau, tout en faisant la publicité du dispositif, mais n'ont pas encore vocation à devenir une publication à part entière. En 1641, face au succès du Bureau de l'île de la Cité, une deuxième adresse ouvre ses portes sous les galeries du Louvre. Après la mort du cardinal Richelieu en 1646, puis de Louis XIII et enfin de Renaudot lui-même en 1653, la diffusion des\textit{ Feuilles} (ou \textit{Liste}, comme elles sont parfois appelées) devient beaucoup plus disparate. Le Bureau d'adresses est confié par le fils Renaudot à des concessionnaires, qui assurent sa subsistance mais ne lui permettent plus de connaître le succès qu'il avait rencontré à ses débuts.

\subsection{La renaissance et l'établissement des\textit{ Affiches} au XVIIIè siècle}

Au milieu du XVIIIè siècle, le format de la petite annonce est ressuscité par plusieurs initiatives individuelles, pour la première fois stables et durables, inspirées du succès du format à l'étranger, notamment en Allemagne. Les annonces font leur apparition dans les titres généralistes, mais également dans des publications spécialisées et dédiées aux petites annonces. La famille Renaudot cède ainsi en 1749 les privilèges de la \textit{Gazette} et du Bureau d'adresses; les deux entrepreneurs qui les récupèrent lancent successivement en 1751 les \textit{Affiches de Paris} puis en 1752 les \textit{Affiches de province}. Le modèle économique des deux publications repose sur l'abonnement: relativement cher, réservé à un lectorat plutôt fortuné (qu'Ulrike Krampl désigne comme "consommateurs éclairés"), il garantit la gratuité du dépôt des annonces par, entre autres, les domestiques en recherche d'un emploi. Face au succès de l'édition parisienne, des versions régionales paraissent, dont celles qui m'intéressent ici: les \textit{Affiches de Lyon} au début des années 1750, puis les \textit{Affiches de Bordeaux} en 1758.
À l'origine bi-hebdomadaires, au format de huit pages in-octavo, les \textit{Affiches de Paris} deviennent quotidiennes à la fin des années 1770. Au total, vers 1780, elles publient 3 000 à 5 000 annonces par mois, pour des tirages à 3000 exemplaires, qui doublent encore dans les dix années suivantes. En 1788, face au rythme des demandes, que le journal ne parvient à soutenir malgré la publication régulière de suppléments de huit ou seize pages, la publication d'une annonce devient payante, au prix d'une livre quatre sols\footcites{kramplPresseAnnoncesParisienne2020}. 

\subsection{Annonces et recherche d'emploi, une apparition plus tardive}


Le contenu des \textit{Affiches}, dès leurs débuts, se caractérise par une grande diversité, et est organisé par rubriques: annonces commerciales évidemment, immobilières (vente ou location) ou mobilières (meubles, chevaux, voitures), publicités (notamment relatives aux parutions des libraires), informations (mariages, enterrements) ou recherche d'informations (objets perdus). Certains titres publient même les programmes des salles de théâtre, les dernières décisions de justice ou le cours des monnaies. La seule catégorie qu'on n'y trouve pas (encore) est celle de la recherche matrimoniale, qui n'investira ce support que plus tard, et plutôt dans des publications spécialisées. 

Les annonces d'emploi, quant à elles, apparaissent à Paris en 1760. Elles sont d'abord majoritairement des offres d'emploi, probablement du fait du caractère encore relativement intimiste des \textit{Affiches} en dehors de son "lectorat éclairé". Au cours des années 1770, les demandes deviennent progressivement majoritaires. Au total, entre 1760 et 1788, 19 000 annonces d'emploi sont publiées, dont trois quarts sont des demandes\footcites{kramplPresseAnnoncesParisienne2020}. 
Pour ce qui est des \textit{Affiches provinciales}, ces chiffres et ces informations relatives à la place des annonces d'emploi dans l'économie du journal n'ont, à ma connaissance, jamais été relevés.


\section{Presse, travail et humanités numériques}

\subsection{Presse, presse d'annonces et travail dans l'historiographie}

Face à l'histoire riche de la presse d'annonces en France depuis le XVIIè siècle, quel intérêt historiographique pour cet objet, quel usage par la discipline historique de ces sources?

Les journaux d'annonces ont d'abord été intégrés à l'histoire plus générale de la presse, notamment à partir de sa massification au XIXè siècle. L'ouvrage de référence sur la question\footcites{feyelAnnonceNouvellePresse2000}, issu de la thèse de Gilles Feyel, présente la particularité de se concentrer sur deux titres: un journal généraliste, la \textit{Gazette}, et les \textit{Affiches}, analysées au prime de la "presse d'information". L'ouvrage de synthèse de Marc Martin\footcites{martinTroisSieclesPublicite1992}, quant à lui, leur consacre un chapitre, mais les inclut dans une histoire plus générale de la publicité.

En tant qu'objet historique à part entière, et en tant que sources permettant de sonder les pratiques et les discours du monde social, la "quatrième page" n'a donc que très récemment intéressé les historiennes et les historiens. Quand elle est mobilisée, c'est principalement au prisme de trois objets, historiques ou sociologiques:  "le marché matrimonial, le marché du travail, et dans une moindre mesure le marché immobilier\footcites{frydmanEcrireHistoirePetites2020}". SI elle sert souvent de source d'appoint ou de contrepoint documentaire, elle reste rarement étudiée pour elle-même.

Dans le cadre de l'étude de la recherche de travail, les petites annonces ont pourtant montré leur potentiel heuristique, notamment lorsqu'il s'agit de s'intéresser aux catégories subalternes ou éloignées du marché traditionnel de l'emploi\footcites{bresseyLookingWorkBlack2010}{alvesRosieRiveterJob2012}. Pour le XVIIIè siècle français, c'est principalement l'historienne Ulrike Krampl qui s'est intéressée à un marché très largement occupé par la domesticité urbaine, et où se trouve surreprésentée une certaine condition ancillaire, plus qualifiée et masculine que la moyenne\footcites{kramplPresseAnnoncesParisienne2020,kramplTravaillerAvecLangues2019}. 

\subsection{Le renouveau des Humanités numériques: de la numérisation des sources...}

Face à cet intérêt limité de la recherche pour les journaux d'annonces, les projets de numérisation et de mise à disposition massive de la presse française et européenne depuis les années 2010 (\textit{Europeana newspaper}, NewsEye, Médias 19, RétroNews) représentent autant d'opportunités de se ressaisir de cet objet, voire peut-être même d'en faire émerger de nouveaux, propices à l'investigation numérique\footcites{pinsonLirePresseXIXe2017}. 


\subsection{...à leur exploitation numérique}

Plus que la simple numérisation, c'est en effet l'interrogation de la presse et de la presse d'annonces avec les outils des humanités numériques qui m'intéressent ici. Ces outils se sont en effet déjà révélés prometteurs pour penser le genre de la presse et du travail: la journée d'études du projet NewsEye qui s'est tenue en 2021 à la Bibliothèque nationale de France portait ainsi sur les femmes dans la presse ancienne numérisée. Les premières publications du projet ANR Numapress, ont également pris pour objet les rapports de sexe à l'aune de la "civilisation du journal", en s'intéressant aux titres de presse en eux-mêmes mais également aux "femmes de presse". Les projets "Gender and Work\footnote{Site du projet: \url{https://www.gaw.hist.uu.se/}}" et "Freedom on the Move\footnote{Site du projet: \url{https://freedomonthemove.org/}}", à partir d'objets radicalement différents, ont eux aussi permis de prendre conscience des capacités des méthodes numériques pour faire émerger des individus, des paroles et des pratiques jusque-là marginales\footcites{laiteEmmetInchSmall2020} dans la "grande histoire". 


\section{Corpus et méthodologie}


En prenant en compte ces différentes pistes amenées par l'histoire du travail, l'histoire du genre, l'histoire de la presse et les humanités numériques, ce projet vise à faire l'histoire des domestiques dans la presse d'annonces de trois villes françaises entre le milieu du XVIIIè et le début du XIXè siècle.  

À partir des cas de Paris, Lyon et Bordeaux, trois espaces urbains mais à la composition socio-économique différente sous l'Ancien Régime, il s'agit ici d'interroger à la fois le phénomène de massification de l'annonce d'emploi au XVIIIè siècle à l'aune d'outils quantitatifs, la composition socio-démographique, et notamment genrée, de la population qui publie des annonces et le contenu-même des annonces grâce aux méthodes du traitement computationnel des langues. Il s'agira ensuite de comparer les résultats obtenus avec les acquis de l'historiographie classique de la domesticité, mais également avec ceux, plus récents, de l'histoire du travail dans les annonces. 

Le corpus, dont le processus de constitution sera détaillé dans le chapitre 1 (et dont la composition précise peut être retrouvée en annexe), est donc formé de trois titres de presse d'annonces, les \textit{Affiches de Paris}, les \textit{Affiches de Lyon} et les \textit{Affiches de Bordeaux}, dont plusieurs milliers de numéros ont pu être récupérés.

La méthodologie a principalement reposé sur l'élaboration de scripts Python à l'aide de librairies de science des données (pandas, numpy) et de traitement automatique des langues (spacy, gensim, BERTopic, Top2Vec). Pour certaines parties du corpus, j'ai entraîné un modèle d'apprentissage, d'extraction et de catégorisation d'entités, qui m'a permis d'automatiser en partie le processus de segmentation des annonces.

Une première partie du mémoire s'intéressera au processus d'élaboration du corpus, qui s'est découpé en plusieurs étapes: récupération des titres numérisés sur des bibliothèques en ligne, océrisation des numéros, segmentation des textes obtenus, sélection des seules annonces d'emploi, puis correction et pré-traitement de celles-ci. Ensuite, une première exploration des annonces s'intéressera aux données socio-démographiques qui peuvent être extraites du corpus, et à ce que ces statistiques nous disent de la représentativité ou des disparités entre population générale et population des annonces. Puis c'est le corps-même de l'annonce qui fera l'objet d'un examen: les compétences mises en avant, les qualités invoquées et les emplois recherchés par les domestiques sont autant d'indices de l'état du marché du travail domestique. Les adresses et intermédiaires mobilisés à la fin des annonces seront ensuite investigués dans la quatrième partie, en vue de replacer le support du journal parmi l'ensemble des réseaux de cooptation et d'intermédiation qui co-existent au XVIIIè siècle. Enfin, la cinquième et ultime partie se penchera sur les apports des méthodes de \textit{topic modeling} pour la compréhension globale du corpus.
